\documentclass[12pt, a4paper]{article}

\usepackage{geometry}
\geometry{verbose,a4paper,tmargin=2cm,bmargin=2cm,lmargin=2.5cm,rmargin=1.5cm}
 
\usepackage[english]{babel}
\usepackage{amsmath}
\usepackage{mathtools}
\usepackage{cases}
\usepackage{hyperref}

\title{SIR models for the spread of COVID-19}
\date{\today}
\author{
	Maksim Gordienko \\
	Alexander Sysoev
}

\begin{document}
	\maketitle
	
	\section{Introduction}
	Write smth about covid-19, motivation, SIR model, Kermack
	and McKendrick etc.
	
	\section{Mathematical model of epidemics}
	Let's consider a group of $N$ people and classify them into these types:
	\begin{itemize}
		\item \textbf{S} -- Susceptible
		\item \textbf{I} -- Infected
		\item \textbf{R} -- Recovered
	\end{itemize}

	This is called the SIR model for the spread of epidemic diseases that describes the changes in numbers of the three types of individuals. We denote the number of susceptible persons by $s(t)$, the number of infected individuals by $i(t)$ and the number of recovered people by $r(t)$. The time $t$ is measured in days. Also we suppose that each person is in contact with $m$ persons per day average. Hence the number of contacts between the susceptible and infected people becomes $\frac{m}{N} i(t) s(t)$. If we set the probability of infection for each contact as $p$ then the number of newly infected individuals within $\Delta t$ days becomes
	\begin{equation}
		\frac{m}{N} i(t) s(t) p \Delta t
	\end{equation}
	in total. Let $\beta = mp$, the number of non-infected people \textbf{(S)} from the $t$-th day to $ (t + \Delta t) $-th day changes as
	\begin{equation}
		s(t + \Delta t) - s(t) = - \frac{\beta}{N} s(t) i(t) \Delta t
	\end{equation}
	When $\Delta t \rightarrow 0 $, we can rewrite it as differential equation
	\begin{equation}
		\frac{ds(t)}{dt} = - \frac{\beta}{N} s(t) i(t)
	\end{equation}
	Meanwhile, the infected individuals recover at a removal rate $ \gamma $ per day. Subsequently, increase in the number of the recovered persons becomes
	\begin{equation}
		\frac{dr(t)}{dt} = \gamma i(t)
	\end{equation}
	Respectively, $ \dfrac{1}{\gamma} $ is the expected duration of infection. Also the number of recovered people includes the amount of deceased persons because they cannot possibly infect others.
	
	\newpage
	
	As well as the total number of individuals is $N$, we can express $i(t)$
	\begin{equation}
		i(t) = N - s(t) - r(t)
	\end{equation}
	The change in the number of infected people can be written as
	\begin{equation}
		\frac{di(t)}{dt} = - \frac{ds(t)}{dt} - \frac{dr(t)}{dt} = \frac{\beta}{N} s(t) i(t) - \gamma i(t) = \frac{\beta}{N} (s(t) - \gamma) i(t)
	\end{equation}
	Gathering the equations above, we have
	\begin{numcases}{}
		\frac{ds(t)}{dt} = - \frac{\beta}{N} s(t) i(t) \label{eq:dsdt} \\
		\frac{di(t)}{dt} = (\frac{\beta}{N} s(t) - \gamma) i(t) \label{eq:didt} \\
		\frac{dr(t)}{dt} = \gamma i(t) \label{eq:drdt}
	\end{numcases}
	As initial conditions, we use
	\begin{equation} \label{eq:init_cond}
		\begin{dcases}
			s(0) = N_1 \\
			i(0) = N_2 \\
			r(0) = 0
		\end{dcases}
	\end{equation}
	If we introduce the following transformations of variables
	\[
		\tilde{s}(t) = \frac{s(t)}{N}, \tilde{i}(t) = \frac{i(t)}{N}, \tilde{r}(t) = \frac{r(t)}{N}, \tilde{t} = \beta t
	\]
	Then system of the equations becomes
	\begin{equation}
		\begin{dcases}
			\frac{d \tilde{s} (t)}{d \tilde{t}} = - \tilde{s}(t) \tilde{i}(t) \\
			\frac{d \tilde{i}(t)}{d \tilde{t}} = (\tilde{s}(t) - \frac{1}{R_0}) \tilde{i}(t) \\
			\frac{d \tilde{r}(t)}{d \tilde{t}} = \frac{1}{R_0} \tilde{i}(t)
		\end{dcases}
	\end{equation}
	The number $R_0 = \dfrac{\beta}{\gamma}$ is known as the basic reproduction number. The number of infected people increases when $R_0 > 1$ and decreases when $R_0 < 1$.
	
	\section{Analytical solution of SIR model}
	When model is defined, we can solve the system of equations \eqref{eq:dsdt} -- \eqref{eq:drdt}. For simplicity forget about function arguments. Firstly, rewrite the equation \eqref{eq:dsdt} as
	\begin{equation} \label{eq:dsdt_rewritten}
		i = - \frac{1}{\tilde{\beta}} (\frac{s'}{s})
	\end{equation}
	where $\tilde{\beta} = \dfrac{\beta}{N}$. Then differentiate both sides
	\begin{equation} \label{eq:di_1}
		i' = - \frac{1}{\tilde{\beta}} (- \frac{{s'}^2}{s^2} + \frac{s''}{s})
	\end{equation}
	Next insert the equation \eqref{eq:dsdt_rewritten} into \eqref{eq:didt}
	\begin{equation} \label{eq:di_2}
		i' = -(\tilde{\beta} s - \gamma) \frac{1}{\tilde{\beta}} (\frac{s'}{s})
	\end{equation}

	\newpage
	
	Comparing equations \eqref{eq:di_1} and \eqref{eq:di_2} we have
	\begin{equation} \label{eq:dsdt_combined}
		s \frac{d^2 s}{dt^2} - (\frac{ds}{dt})^2 + (\gamma - \tilde{\beta} s) s \frac{ds}{dt} = 0
	\end{equation}
	Now we introduce the following function
	\begin{equation} \label{eq:phi_intro}
		\phi = \frac{dt}{ds}
	\end{equation}
	Afterward \eqref{eq:dsdt_combined} becomes
	\begin{equation}
		\frac{d \phi}{ds} + \frac{\phi}{s} = (\gamma - \tilde{\beta} s)\phi^2
	\end{equation}
	This is a Bernoulli differential equation. Divide both parts by $ \phi^2 $
	\begin{equation}
		\frac{\phi'}{\phi^2} + \frac{1}{\phi s} = \gamma - \tilde{\beta} s
	\end{equation}
	Then make a substitution
	\begin{equation}
		z = \frac{1}{\phi}, z' = -\frac{\phi'}{\phi^2}
	\end{equation}
	Now solve the first-order linear ordinary equation
	\begin{equation}
		-z' + \frac{z}{s} = \gamma - \tilde{\beta} s 
	\end{equation}
	with general solution where $C$ is constant
	\begin{equation}
		z = -\gamma s \ln{s} + \tilde{\beta} s^2 + Cs
	\end{equation}
	Returning back to $\phi$
	\begin{equation} \label{eq:phi_res}
		\phi = \frac{1}{s(C - \gamma \ln{s} + \tilde{\beta} s)}
	\end{equation}
	From the relation of inverse function in equation \eqref{eq:phi_intro}, we have
	\begin{equation}
		\frac{1}{\phi} = \frac{ds}{dt}
	\end{equation}
	Using equation \eqref{eq:dsdt}, we obtain
	\begin{equation} \label{eq:it_def}
		i(t) = -\frac{1}{\tilde{\beta}} (C - \gamma \ln{s(t)} + \tilde{\beta} s(t))
	\end{equation}
	Moreover, from equations \eqref{eq:dsdt} and \eqref{eq:drdt}, we get
	\begin{equation}
		\frac{dr}{dt} = -\frac{\gamma}{\tilde{\beta}} (\frac{s'}{s})
	\end{equation}
	Subsequently, the relation between s(t) and r(t) becomes
	\begin{equation} \label{eq:rt_def}
		r(t) = -\frac{\gamma}{\tilde{\beta}} \ln{\frac{s(t)}{C_1}}
	\end{equation}
	where $C_1$ is a constant. According to our initial conditions \eqref{eq:init_cond} 
	\begin{equation}
		C_1 = N_1
	\end{equation}

	\newpage
	
	From the relation $s(0) + i(0) + r(0) = N$, we have
	\begin{equation}
		C = -\tilde{\beta}N + \gamma \ln{N_1}
	\end{equation}
	If we substitute $C$ into equation \eqref{eq:phi_res}, we obtain
	\begin{equation}
		\frac{dt}{ds} = \frac{1}{s(-\tilde{\beta}N - \gamma \ln{\dfrac{s}{N_1} + \tilde{\beta}s})}
	\end{equation}
	Integrate this and express $t$ as a function of $s$
	\begin{equation} \label{eq:t_eps}
		t = \int_{s(0)}^{s(t)} \frac{d \varepsilon}{\varepsilon(-\tilde{\beta}N - \gamma \ln{\dfrac{\varepsilon}{N_1}} + \tilde{\beta} \varepsilon)}
	\end{equation}
	For convenience, we change a variable
	\begin{equation}
		\xi = \frac{\varepsilon}{N_1}
	\end{equation}
	Rewrite the equation \eqref{eq:t_eps}
	\begin{equation}
		t(s) = \int_{1}^{\frac{s(t)}{N_1}} \frac{d \xi}{\xi(-\beta - \gamma \ln{\xi} + \beta \xi \frac{N_1}{N})}		
	\end{equation}
	Now we can calculate $t(s)$ using numerical integration with a small step size and $s(t)$ as a parameter like that
	\begin{equation}
		\int_{a}^{b} f(\xi) \,d\xi \simeq \sum_{i=0}^{n-1} f(a + ih)h,~ \text{where} ~ h = \frac{b-a}{n}
	\end{equation}
	If $t(s)$ is obtained, then $i(t)$ and $r(t)$ can be calculated from equations \eqref{eq:it_def} and \eqref{eq:rt_def} respectively.
	
	\newpage

	\section{SIR Model with natural deaths and births}

	Let's rewrite our system of equations \eqref{eq:dsdt} -- \eqref{eq:drdt} with an addition of the death and birth processes. Consider these rates are equal to each other, we introduce new variable $D$:
	\begin{numcases}{}
		\frac{ds(t)}{dt} = - \frac{\beta}{N} s(t) i(t) + D(N - s(t)) \label{eq:death_s} \\
		\frac{di(t)}{dt} = (\frac{\beta}{N} s(t) - \gamma - D) i(t) \label{eq:death_i} \\
		\frac{dr(t)}{dt} = \gamma i(t) - Dr(t) \label{eq:death_r}
	\end{numcases}
	As we did before, we rewrite it into one equation:
	\begin{equation} \label{eq:re_death_s}
		i = \frac{s' - D(N- s)}{-\tilde{\beta}s}
	\end{equation}
	Again we differentiate both sides (with $\tilde{\beta} = \frac{\beta}{N}$ substitution):
	\begin{equation}
		i' = -\frac{1}{\tilde{\beta}}(\frac{(s'' + D s')s - (s' + D s - D N)s'}{s^2}),
	\end{equation}
	\begin{equation} \label{eq:re_death_s_diff}	
		i' = -\frac{1}{\tilde{\beta}}(\frac{s''}{s} - \frac{s'^2 - D N s'}{s^2})
	\end{equation} 
	Now we can insert \eqref{eq:re_death_s} into \eqref{eq:death_i}:
	\begin{equation}
		i' = (\tilde{\beta} s - \gamma - D)(\frac{s' - D(N- s)}{-\tilde{\beta}s})
	\end{equation}
	\begin{equation} \label{eq:re_death_i_sub}
		i' = -\frac{1}{\tilde{\beta}s}(\tilde{\beta}s - \gamma - D)(s' + D s - D N)
	\end{equation}
	And comparing \eqref{eq:re_death_i_sub} and \eqref{eq:re_death_s_diff} we get next equation:
	\begin{equation} 
		s s'' - s'^2 +(-\tilde{\beta}s^2 + (\gamma + D)s + D N)s' - \tilde{\beta}D s^3 - (\tilde{\beta}D N + D(\gamma + D))s^2 - (\gamma + D)D N s = 0
	\end{equation}
	We can see that coefficients at $s^3, s^2$ and $s$ are $consts$ so we substitute them with \\ 
	$A = -\tilde{\beta}D, B = - (\tilde{\beta}D N + D(\gamma + D)),  C = - (\gamma + D)D N$:
	\begin{equation} \label{eq:death_main}
		s s'' - s'^2 +(-\tilde{\beta}s^2 + (\gamma + D)s + D N)s' +A s^3 + B s^2 + C s = 0
	\end{equation}
	Now we introduce a function $\phi = \frac{dt}{ds}$, so the equation will be:
	\begin{equation} \label{eq:death_main_phi}
		-\phi' - \frac{\phi}{s} +(-\tilde{\beta}s^2 + (\gamma + D)s + D N)\frac{\phi^2}{s} + (A s^2 + B s + C)\phi^3 = 0
	\end{equation}
	And by substitution $v = \phi s, v' = \phi's + \phi$ the following equation occur:
	\begin{equation} \label{eq:death_main_v}
		v' +(-\tilde{\beta} + \frac{(\gamma + D)}{s} + \frac{D N}{s^2})v^2 + (A + \frac{B}{s} + \frac{C}{s^2})v^3 = 0
	\end{equation}

	\newpage
	Then we assume these three substitutions:
	\begin{numcases}{}
		r(s) = \frac{1}{v(s)}, \\
		\xi(s) = (\tilde{\beta} - \frac{(\gamma + D)}{s} - \frac{D N}{s^2}), \\
		\eta(s) = -(A + \frac{B}{s} + \frac{C}{s^2})
	\end{numcases}
	and we get Abel-type first-order second-kind equation:
	\begin{equation}
		r r' = \xi(s)r + \eta(s)
	\end{equation}

	\newpage
	
	\section{Modeling COVID-19 spread}
	In order to create lifelike COVID-19 pandemic model we need to introduce more parameters and consider spread cases.
	\begin{itemize}
		\item \textbf{$S$} -- Susceptible. Includes all individuals who are not infected but are susceptible to contract the disease.
		\item \textbf{$E$} -- Exposed. Includes all persons who are exposed to infection, but not yet infections. Some of them might fall ill and some may not. We introduced that component due to incubation period of COVID-19 pandemic (approximated 5-6 days by WHO).
		\item \textbf{$I_s$} -- Symptomatic infected. It includes all the individuals who were symptomatic and infectious. They have approached some health-care facility, but have not yet been quarantined. This compartment was brought into the picture considering concerning news from worst-affected counties like Italy and the U.S. of the hospitals and health-care centers getting filled up very fast during this pandemic. A significant portion of the infected individuals may not be quarantined in case of the health-care system collapses.
		\item \textbf{$I_{as}$} -- Asymptomatic infected. It includes all those people who are affected but asymptomatic before they either recover or die or get permanently disabled. This compartment was formalized keeping after
		reports of various authentic studies claiming that around 30-40 percent of the infected individuals remain
		asymptomatic.
		
	\end{itemize}
	
	
	
	
\end{document}